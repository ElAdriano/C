\documentclass[10pt,oneside]{article}
\usepackage{polski}
\usepackage{fancyhdr} %numerowanie i toppery
\usepackage{lastpage} %ostatnia strona
\usepackage{indentfirst} %wcięcie
\usepackage{tikz}

\title{Specyfikacja funkcjonalna - BieszczadyTour}
\author{Adrian Bączek}
\date{6.11.2019}

\pagestyle{fancy}
\fancyhf{}

\begin{document}
\maketitle

\rfoot{
	\begin{flushright}
		v1.0
	\end{flushright}
}

\thispagestyle{fancy}

\newpage

\rfoot{
	\begin{center}
		Strona \thepage \hspace{1pt} z \pageref{LastPage}
	\end{center}
}

\section{Wstęp}
\subsection{Cel dokumentu}
Celem dokumentu jest klarowne i~przejrzyste nakreślenie 
odbiorcy programu sposobu korzystania z oprogramowania.
\subsection{Cel projektu}
Jako cel projektu obrany został program wyznaczający najlepszą, możliwą do przebycia, 
trasę na podstawie informacji podanych przez użytkownika.
\subsection{Użytkownik końcowy}
Oprogramowanie zostało stworzone z dedykacją dla osób ciekawych świata 
i~jego wyzwań, by mogli oni tak długo czerpać radość z natury jak tylko zapragną.

\section{Uruchamianie programu}
Uruchomienie programu nastąpi po wywołaniu komendy : $$run.bat~plik\_konfiguracyjny~ID\_miejsca~miejsca\_podrozy$$
Jako argumenty wywołania skryptu podajemy ścieżkę do pliku konfiguracyjnego, identyfikator miejsca 
z którego chcemy rozpocząć wędrówkę oraz ścieżkę do pliku z wybranymi miejscami podróży (argument opcjonalny - w przypadku jego braku program wyznaczy najlepszą trasę nie uzględniając priorytetu poszczególnych punktów).

Przykładowa komenda uruchamiająca program:
\newline
run.bat~C:$\backslash$Users$\backslash$User$\backslash$Desktop$\backslash$trasy.txt~A~C:$\backslash$Users$\backslash$User$\backslash$Desktop$\backslash$do\_odwiedzenia.txt

\section{Dane programu}
Format danych wejściowych oraz wyjściowych
\subsection{Dane wejściowe}
Użytkownik podaje jako argumenty uruchomienia ścieżkę do pliku konfiguracyjnego, identyfikator miejsca startowego oraz opcjonalnie ścieżkę do pliku z~wybranymi przez niego miejscami podróży.
Do poprawnego działania programu wymagany jest poniższy format danych wejściowych :
\begin{itemize}
	\item Plik konfiguracyjny \newline W pliku powinny znale\'zć się odpowiednio oznaczone 2 listy informujące o~punktach szlaku turystycznego i połączeniach między nimi.
	\begin{itemize}
		\item Pierwszy zestaw danych oznaczony jako ,,\#\#\# Miejsca podróży" należy podać w poniższym formacie : \newline
		Lp. $\vert$ ID\_miejsca $\vert$ Nazwa miejsca $\vert$ Opis miejsca $\vert$
		\newline \newline
		\underline{Przykład}
		\newline \newline
		Lp.$\vert$ ID\_miejsca $\vert$ Nazwa miejsca $\vert$ Opis miejsca $\vert$ \newline
		1. $\vert$ A $\vert$ Koliba Studencka Politechniki Warszawskiej $\vert$ Baza wypadowa (miejsce rozpoczęcia wędrówki) $\vert$ \newline
		2. $\vert$ B $\vert$ Jawornik $\vert$ Szczyt (1021) $\vert$ \newline
		3. $\vert$ C $\vert$ Rabia Skała $\vert$ Szczyt (1199) $\vert$ \newline
		4. $\vert$ D $\vert$ Dziurkowiec $\vert$ Szczyt (1189) $\vert$ \newline
		5. $\vert$ E $\vert$ Okrąglik $\vert$ Szczyt (1101) $\vert$ \newline
		6. $\vert$ F $\vert$ Fereczata $\vert$ Szczyt (1102) $\vert$ \newline
		
		\item Druga lista oznaczona jako ,,\#\#\# Czas przejścia" powinna zostać podana w następującym formacie : \newline
		Lp. $\vert$ ID\_miejsca\_początkowego (S) $\vert$ ID\_miejsca\_końcowego (E) $\vert$ Czas S $\to$ E $\vert$ Czas E $\to$ S $\vert$ Jednorazowa opłata za przejście trasą (zł) $\vert$
		\newline \newline
		\underline{Przykład}
		\newline \newline
		Lp. $\vert$ ID\_miejsca\_początkowego (S) $\vert$ ID\_miejsca\_końcowego (E) $\vert$ Czas S $\to$ E $\vert$ Czas E $\to$ S $\vert$ Jednorazowa opłata za przejście trasą (zł) $\vert$ \newline
		1. $\vert$ A $\vert$ B $\vert$ 2:00 $\vert$ 3:00 $\vert$ -- $\vert$ \newline
		2. $\vert$ A $\vert$ C $\vert$ 4:00 $\vert$ 4:30 $\vert$ 5 $\vert$ \newline
		3. $\vert$ B $\vert$ C $\vert$ 1:30 $\vert$ 1:30 $\vert$ -- $\vert$ \newline
		4. $\vert$ B $\vert$ D $\vert$ 1:00 $\vert$ 1:30 $\vert$ -- $\vert$ \newline
		5. $\vert$ B $\vert$ E $\vert$ 1:00 $\vert$ 1:30 $\vert$ -- $\vert$ \newline
		6. $\vert$ C $\vert$ D $\vert$ 3:00 $\vert$ 2:00 $\vert$ -- $\vert$ \newline
		7. $\vert$ D $\vert$ F $\vert$ 4:00 $\vert$ 3:00 $\vert$ -- $\vert$ \newline
		8. $\vert$ E $\vert$ F $\vert$ 0:30 $\vert$ 0:30 $\vert$ -- $\vert$ \newline
	\end{itemize}
	Przy czym czas (S)$\to$(E) oraz (E)$\to$(S) podany jest w formacie GG:MM.
	
	\item ID\_miejsca
	\begin{itemize}
		\item Identyfikator miejsca startowego powinien odpowiadać konkretnej, wybranej przez użytkownika lokalizacji z pliku konfiguracyjnego.
		\newline \newline
		\underline{Przykład}
		\newline \newline
		A
	\end{itemize}
	
	\item Plik z wybranymi miejscami podróży
	\begin{itemize}
		\item Opcjonalny plik (oznaczony wewnątrz jako ,,\#\#\# Wybrane miejsca podróży")~z wybranymi miejscami wycieczki podany powinien być w~formacie : \newline
		Lp. $\vert$ ID\_miejsca $\vert$
		\newline \newline
		\underline{Przykład}
		\newline \newline
		1. $\vert$ A $\vert$ \newline
		2. $\vert$ B $\vert$ \newline
		3. $\vert$ E $\vert$ \newline
	\end{itemize}
\end{itemize}

\subsection{Dane wyjściowe}
Program na podstawie poprawnych danych wyznaczy optymalną trasę, po czym wyeksportuje wynik do pliku "tour.txt".
Zawartość pliku wynikowego przedstawiona zostanie w postaci listy miejsc zaplanowanych do odwiedzenia oraz krótkiego oszacowania czasu i kosztu podróży.
\newline \newline
\underline{Przykład}
\newline \newline
Koliba Studencka Politechniki Warszawskiej \newline
$\to$ Jawornik \newline
$\to$ Dziurkowiec \newline
$\to$ Rabia Skała \newline
$\to$ Jawornik \newline
$\to$ Okrąglik \newline
$\to$ Fereczata \newline
$\to$ Okrąglik \newline
$\to$ Jawornik \newline
$\to$ Koliba Studencka Politechniki Warszawskiej \newline
Czas: 13 godzin 0 minut \newline
Koszt: 0 zł \newline


\section{Scenariusz przebiegu działania programu}
Scenariusz korzystania z programu
\subsection{Scenariusz Ogólny}
\begin{itemize}
	\item Uruchomienie programu z linii komend z odpowiednimi argumentami.
	\item Przeprowadzenie odpowiednich operacji w celu wyznaczenia optymalnej trasy podróży.
	\item Zapisanie trasy do pliku wynikowego w odpowiednim formacie.
	\item Zakończenie programu.
\end{itemize}

\subsection{Scenariusz Szczegółowy}
\begin{itemize}
	\item Uruchomienie programu z linii polecenia przy podaniu argumentów wejściowych.
	\item Walidacja danych wejściowych przez wewnętrzne walidatory programu - w przypadku błędnych danych zwrócony zostaje odpowiedni komunikat, po czym program zakończy działanie.
	\item Program na podstawie sprawdzonych danych przeprowadza odpowiednie operacje wyznaczając optymalną trasę. W trakcie procesu walidowana będzie możliwość dotarcia do wybranych punktów podróży. W przypadku jej braku, zwrócony zostanie właściwy dla sytuacji komunikat, następnie program zakończy działanie.
	\item Program wypisze otrzymane rozwiązanie do pliku wynikowego.
	\item Zakończenie działania programu.
\end{itemize}

\subsection{Opis sytuacji wyjątkowych}
\begin{description}
	\item[Błąd formatu danych wejściowych] - rozpoznany zostaje przez wewnętrzny walidator programu. W przypadku wykrycia nieprawidłowości zwrócony zostanie kod błędu formatu wejścia (oznaczony kodem 1). Następnie program zakończy działanie.
	\item[Błąd niespójności danych] - błąd owy zostanie zidentyfikowany przez wewnętrzny walidator programu. W~przypadku jego wystąpienia zwrócony zostanie kod błędu (oznaczony kodem 2). Następnie program zakończy działanie.
	\item[Błąd ID miejsca startowego] - niezgodność rozpoznana przez wewnętrzny walidator programu. W przypadku niepoprawnego ID miejsca startowego zwrócony zostanie kod błędu miejsca startowego (oznaczony kodem 3) i program zakończy działanie. 
\end{description}

\section{Testowanie}
Program testowany będzie pod względem trzech przypadków danych wejściowych :
\begin{description}
	\item[Przypadek błędnego formatu danych wejściowych] - testowany będzie walidator formatu danych pobranych z plików podanych jako argumenty wywołania. Testowanie odbędzie się poprzez podanie danych w poprawnym formacie i błędnym. Oczekiwanym wynikiem będzie odpowiednio brak kodu błędu (oznaczony przez 0) oraz 1.
	
	\item[Przypadek błędu niespójności danych] - dane pobrane z plików wejściowych zostaną przewalidowane przez wewnętrzne mechanizmy programu pod względem spójności tj. sprawdzona zostanie możliwość dotarcia do każdego, wybranego przez użytkownika, punktu podróży. W celu sprawdzenia poprawności walidatorów, argumentami wywołania programu będą pliki z spójnymi oraz niespójnymi danymi. Oczekiwanymi wynikami zwróconymi przez walidatory będą kody odpowiednio 0 i 2.
	
	\item[Przypadek błędu ID miejsca startowego] - poddanym testowaniu zostanie walidator odpowiedzialny za poprawność ID miejsca startowego. W tym celu program zostanie uruchomiony z istniejącym w podanych plikach identyfikatorem oraz nie istniejącym. Oczekiwanymi wynikami będzie brak błędu oraz 3.
\end{description}
Ponadto program testowany będzie pod względem poprawności znalezionej trasy. W tym celu generowane zostaną pliki z danymi dla maksymalnie 7 punktów możliwymi do odwiedzenia. Następnie odnalezione będą wszystkie możliwe ścieżki. W końcowym etapie walidacji nastąpi weryfikacja czy zaproponowana przez metodę szukającą ścieżka jest równoważna jednej z wcześniej wygenerowanych.

\end{document}
